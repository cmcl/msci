\documentclass[11pt]{article}
\usepackage{gensymb}
% Set the paper size to A4
\setlength{\paperheight}{297mm}
\setlength{\paperwidth}{210mm}
% Define commands which allow the width and height of the text
% to be specified. Centre the text on the page.
\newcommand{\settextwidth}[1]{
\setlength{\textwidth}{#1}
\setlength{\oddsidemargin}{\paperwidth}
\addtolength{\oddsidemargin}{-\textwidth}
\setlength{\oddsidemargin}{0.5\oddsidemargin}
\addtolength{\oddsidemargin}{-1in}
}
\newcommand{\settextheight}[1]{
\setlength{\textheight}{#1}
\setlength{\headheight}{0mm}
\setlength{\headsep}{0mm}
\setlength{\topmargin}{\paperheight}
\addtolength{\topmargin}{-\textheight}
\setlength{\topmargin}{0.5\topmargin}
\addtolength{\topmargin}{-1in}
}
% Top and bottom margins should be 15mm, so text height is 267mm
\settextheight{267mm}
% Left and right margins should be 25mm, so text width is 160mm
\settextwidth{160mm}
% No page number
\pagestyle{empty}
\setlength{\parskip}{6pt}
\begin{document}
\begin{tabular*}{160mm}{|l@{\extracolsep{\fill}}c@{\extracolsep{0pt}}|l|}
\hline
Title:
Lolliproc: to Concurrency from Classical Linear Logic via Curry-Howard and
Control

& &
Name:
Craig McLaughlin

\\ \hline
Author[s]:
Karl Mazurak and Steve Zdancewic

& &
Student number:


\\ \hline
\end{tabular*}

\begin{center}
\begin{minipage}{140mm}
\small
Abstract: %
There exist several approaches to utilising the power of linear logic in type
systems and programming languages, from full linear logic incorporating
classical components to the intuitionistic fragment. However, all approaches
suffer from either poor integration of concurrency support or a language which
is not in tune with mainstream functional programming. The current work
presents a language, Lolliproc, to fill the void which provides a surface
syntax familiar to the function programmer and simple primitives for
concurrency which hide the underlying process calculus from the programmer.
The language semantics ensure deadlock-freedom and absence of race conditions.
This work serves as an interesting base on which to build lambda calculus
based type systems which utilise linearity in order to represent interprocess
communcation via session types. Exploring extensions to the language to
increase its expressiveness could have far reaching implications for the way
in which communication protocols are designed and implemented.

\end{minipage}
\end{center}

Previous work on integrating the features of linear logic into type systems
and programming languages have focussed on three strands. First, construction
of functional programming languages that are similiar to the mainstream were
developed using the intuitionistic portion of linear logic (as opposed to
classical linear logic) which do not have a strong connection between the
linearity present in the calculi and the concurrency primitives provided.
The second approach harnessed the full power of classical linear logic within
the sequent calculus but it is difficult to map this system back to
conventional functional languages due to the symmetry inherent in the language
operations. The third approach invovles a natural deduction system for linear
logic, which has stronger connection to mainstream functional languages but
the semantics do not provide a simple view of the concurrency implications of
the logic and rely on techniques to simulate concurrent evaluation.

The current work presents a language based on natural deduction that bridges
the gap in the previous works. Specifically, a concurrent functional
programming language is presented as the source level syntax backed up with an
underlying process calculus to provide the concurrency mechanisms. The details
of processes are hidden from the programmer using novel adaptions on control
operators. The language encodes classical linear logic adding double negation
elimination as a language primitive. The operational semantics of the linear
lambda calculus are standard with the exception of performing reduction of
evaluation contexts at the process level.

The control operators facilitate spawning child processes and waiting in the
parent process for the child to complete execution. When a process is spawned
the parent and child are each given an endpoint of a channel (sending end is
given to the child). A closed channel token indicates that communication has
finished and that the child process can terminate (provided all linear
resources have been used). Intuition of control operators is that a
continuation passed to the spawning operator is a function executing a
protocol.

 intuition of go is that it accepts a function which is then spawned as a
child process which must execute as the protocol specified by the function
type; a yield then blocks the parent to receive information from the child
then both child and parent continue executing the protocol (this is
essentially an encoding of session types)
 The go operator defined to take a function which describes a protocol which
a child process will execute and the dual type of the protocol is returned to
the parent process on which to yield
 Annotate go operations with protocol type for process level semantics
 No new family of types are introduced to represent channels; regular linear
functions serve this purpose, consequence: bidirectional communication from
a seemingly unidirectional language
 Underneath source interface (linear lambda calculus) is process language (a
la pi calculus) but programmer isn't exposed to processes
 Seemingly only permits communication in one direction yet can mimic
bidirectional communication using session types -- elaboration needed
 Future example enodes sub-computations offloaded to child processes by
encapsulating work inside a function (the authors call it a ``thunk'')
 Linking endpoints without having to yield in the parent is possible with a
clever definition of a function which passes a completion token on to the
parent process; the first child spawn ends up with linked endpoints,
communication has thus been established
 An elaborate example shows how to reserve the direction of a channel rather
than the spawning processing having the source it gets the sink, additionally
the example shows that while values may cycle between processes there are
never any cycles in the communication; essential for soundness proof
 Processes only appear at runtime; no need to implement process typing rules
in typechecker (really? Is that what they are saying here pg.13)




ROF
 Type soundness (progress and preservation) implies deadlock-freedom and
correct session typing
 Strong normalisation: all reduction sequences reach a normal form meaning
no non-terminating programs
 Confluence imposing the same normal form on two reduction sequences starting
from the same redex prohibits race conditions
 Expression typing rules ``follow natural-deduction presentation of
intuitionistic linear calculi''
 Both term and channel contexts are linear (no weakening or contraction)
 Progress only considered for processes
 Coq proofs (need to deal with context permutation since they are linear) but
are not completed; up to preservation and progress on processes (graph based
reasoning the issue)
 Parallel processes form an acyclic communication graph, the restriction of
one channel between two halves of a parallel composition is essential
 Used a weight assignment to processes (and channels) to sketch proof of
strong normalisation
 Show confluence from strong normalisation and the diamond property (local
confluence)


RCL
RTC
RFW
* Unrestricted types, recursive types, nondeterminism
* Polymorphism (extension to System F-pop)
* Inclusion of protocol kinds into the kinds of System F-pop rather than the
syntactic constructs given in Lolliproc
  - has interesting notions e.g. sending (quantified) types between processes
* Claimed strong connection between channel endpoints definition and focused
proofs
POC
* Two processes can share only one channel with the send endpoint in one of
the processes and the receive in the other but is needed for type soundness
WIL
* Don't understand how go is both an identity function (therefore accepting
and returning the same function type) and a primitive which returns the dual
type of the protocol (I cannot get the dual from negating the function type)
* Can the processes exist independently? That is, can we construct two
separate process and link the endpoints? This is done with public access
capabilities in other works; what would the equivalent be here?
* How can linearly typed functions (powmod in alice+bob example) be used more
than once?

\end{document}
