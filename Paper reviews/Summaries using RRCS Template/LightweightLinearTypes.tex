\documentclass[11pt]{article}
\usepackage{gensymb}
% Set the paper size to A4
\setlength{\paperheight}{297mm}
\setlength{\paperwidth}{210mm}
% Define commands which allow the width and height of the text
% to be specified. Centre the text on the page.
\newcommand{\settextwidth}[1]{
\setlength{\textwidth}{#1}
\setlength{\oddsidemargin}{\paperwidth}
\addtolength{\oddsidemargin}{-\textwidth}
\setlength{\oddsidemargin}{0.5\oddsidemargin}
\addtolength{\oddsidemargin}{-1in}
}
\newcommand{\settextheight}[1]{
\setlength{\textheight}{#1}
\setlength{\headheight}{0mm}
\setlength{\headsep}{0mm}
\setlength{\topmargin}{\paperheight}
\addtolength{\topmargin}{-\textheight}
\setlength{\topmargin}{0.5\topmargin}
\addtolength{\topmargin}{-1in}
}
% Top and bottom margins should be 15mm, so text height is 267mm
\settextheight{267mm}
% Left and right margins should be 25mm, so text width is 160mm
\settextwidth{160mm}
% No page number
\pagestyle{empty}
\setlength{\parskip}{6pt}
\begin{document}
\begin{tabular*}{160mm}{|l@{\extracolsep{\fill}}c@{\extracolsep{0pt}}|l|}
\hline
Title:
Lightweight Linear Types in System F\degree

& &
Name:
Craig McLaughlin

\\ \hline
Author[s]:
Karl Mazurak, Jianzhou Zhao, and Steve Zdancewic

& &
Student number:


\\ \hline
\end{tabular*}

\begin{center}
\begin{minipage}{140mm}
\small
Abstract: %
There are various approaches to providing both linear and non-linear access to
resources within a programming language. However, they either hamper the
inclusion of desirable programming features, or do not adequately reflect the
common case. The present paper presents an extension to System F which avoids
these deficiencies by categorising types whilst maintaining semantics
familiar to the base language. Mechanised proofs for type soundness and
parametricity are presented, and a relationship between linear and non-linear
types is defined explicating the semantics of linearity. Extensions to the
system are suggested in order to, among other things, facilitate usage in a
general-purpose functional language. The system can encode a wide variety of
safe interfaces and simulate a range of reference management schemes,
illustrating the applicability of the technique.

\end{minipage}
\end{center}

Previous work on incoporating linearity into a type system has proposed two
approaches. One approach is to consider all types to be linear by default,
and provide a construct (let !, in the literature) for temporary non-linear
access to the resource. The problem with this approach is that non-linear
references and access patterns are more common than linear ones, resulting in
a rather cumbersome programming approach involving a lot of binders. A second
technique is to use type qualifiers (one for linear, one for non-linear) then
all types are a combination of a base (such as Int, or String) and a type
qualifier. However, to allow for desirable programming features such as
polymorphism, the technique requires that all polymorphic functions be
quantified over all type qualifiers, even for simple functions such as plus
(where it could be called on either a linear or non-linear Int). Additionally,
in some situations, there is a relationship between type qualifiers, burdening
the programmer with unnecessary detail and complexity.

WTD
Extend System F by categorising types as being of two different ``kinds''
(linear or non-linear)
Present operational semantics, and proofs of soundness, progress +
preservation, and respects parametricity. Includes a subsumption rule for
kinds allowing non-linear kinds to be treated as linear.
Linearity aware run-time semantics permit one substitution of a value into
a function by separating out uses of values into two categories: conscripted
and discharged. Upon a value's first use as a function argument it is
conscripted then any subsequent use of that value, in context where replacing
it with a free variable would result in a stuck configuration, it is
discharged. The reduction steps are augmented with the set of conscripted
and discharged values during the step.
Some extensions of the language are described to incorporate features similar
to those provided by practical functional programming languages. In addition,
support for programming using linear types is suggested; making use of
syntactic to simplify or eliminate the need for the programmer to thread the
linear references through the program or directly handle existential
quantification.
ROF
An example translation from the current method to a previously proposed method
(let !) is presented but it is noted that translation in the opposite
direction is difficult since, for example, any functions taking an
unrestricted type must now take a ! type and bind it so we can have
unrestricted access, and calls within the function will need to rebind the
variable under !. Note, the let ! typing rule:$ let !x = t_1 in t_2$ requires
$t_1$ to have type $!o$.
RFW
If the recommended extensions do not appear in the mechanisation then that
would be an appropriate future work.
Extensions present algebraic datatypes semantics with constructors with
value argument considered a value.
POC
However, the syntactic sugar description I found to be rather confusing and
the syntactic sugar specialised. What is mentioned is the need to ensure
the syntactic sugar correctly infers types. It appears to be straightforward
but is an additional overhead over the system.
It is not mentioned whether the suggested extensions appear in the
mechanisation or not.
Possible discrepancy in algebraic datatype extension, constructors can expect
``additional type parameters, which act as through [sic] they are
existentially bound'' but the semantics of case indicate they are universally
bound.

\end{document}
