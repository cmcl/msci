\section{Future Work}

Some areas of future work to look into would be the following. The issues
regarding cut elimination suggest there should be greater study into calculi
based on the sequent calculus.

\subsection{Besides Linear}

The ``adoption and focus'' system~\cite{Fahndrich:2002} provides a notion of
aliased access to linear resources. The type system requires special language
features, e.g. to manage capabilities which can be thought as a form of object
liveness mechanism, so GV typing judgements would need to be extended such
that the capabilities can be included in soundness results. An extension to
the translation to CP to incorporate the ``adopt'' and ``focus'' constructs
would also need to be defined. It is not immediately clear how these features
would assist in the management of session typed channels. The ``focus'' rule
(and linearity generally) enforces an invariant on the focussed object
(channel) such that the type is maintained during the focus operation:
\begin{gather*}
  \prooftree
    \begin{array}{l}
      \afsequent{\Delta}{\Gamma}{C}
                {\typedid{e_1}{\Tguard{G}{h}}}
                {\tensor{C_1}{C_2}} \\
      \capsequent{\Delta}{C_1}{G} \\
      \rho~\mbox{fresh} \\
      \afsequent{\Delta}{\Gamma[\typedid{x}{\AFtr{\rho}}]}
                {\tensor{C_2}{\AFcap{\rho}{h}}}
                {\typedid{e_2}{\tau_2}}
                {\tensor{C_3}{\AFcap{\rho}{h}}}
    \end{array}
  \justifies
    \afsequent{\Delta}{\Gamma}{C}
              {\typedid{\LetFocus{x}{e_1}{e_2}}{\tau_2}}
              {\tensor{C_1}{C_3}}
  \using{\mbox{[focus]}}
  \endprooftree
\end{gather*}


The typing judgements include \emph{capabilities} that track aliasing to heap
objects, e.g. $\AFcap{\rho}{h}$ denotes a mapping from a static name $\rho$ to
a heap type $h$, and a variable $x$ may be typed as $\typedid{x}{\AFtr{\rho}}$
indicating that $x$ refers to an object of type $h$. The purpose of the
capabilities is to enforce restrictions on when an object can be
referenced. In other words, if $\AFcap{\rho}{h}$ is in the capabilities at
some program point then $x$ may be used, otherwise such use is prohibited. In
the [focus] rule, evaluating $e_1$ retrieves the object to be focussed. Note
the type of $e_1$ is a ``guarded'' type. The guard $G$ is the static name of
some object which ``adopted''~\footnote{I omit the adopt typing rule for
  brevity.} an object of heap type $h$. If a program point has
$\AFcap{G}{h_1}$ in the capabilities for some $h_1$ then all objects guarded
by $G$ may be used; $\capsequent{\Delta}{C_1}{G}$ denotes containment of a
such a capability in $C_1$. The guarded type may be thought of as policing
manipulation of linear components within $h$. Treating $\otimes$ as the
disjoint union for capability sets, the typing for $e_2$ temporarily removes
the guard $G$ from the set of capabilities, and adds a new variable $x$, and
associated capability, with a type permitting access to the object of type $h$
i.e. a non-guarded type. Removing $G$ will prohibit access to allow objects
guarded by $G$ for the during of $e_2$. The final capabilities after executing
$e_2$ are required to be $\tensor{C_3}{\AFcap{\rho}{h}}$, enforcing that the
type must remain as $h$. The typing of the $\letterm$ expression ensures that
the guard $G$ (contained in $C_1$) is restored.

Such a rule would appear to only benefit recursive session types. Indeed,
focussing on a non-recursive channel would be pointless since one would be
prevented from communicating across it since the focus rule requires
maintenance of the channel type. Further, it would be undesirable to permit
modification of a session type without guaranteeing exclusive access to the
channel since this could lead to invalid communication e.g. trying to
communicate on a closed channel. Additionally, a linear object is consumed
upon being adopted and will be deallocated when its adopter's lifetime ends,
although an alternative semantics could be defined which returned the linear
resource to the context. Moreover, \citeauthor{Fahndrich:2002} do not provide
operational semantics or soundness results for the ``adoption and focus'' type
system.

%% System F-pop and ``adoption and focus''
%%  It would be interesting to see if one could encode the ``adopt'' and
%% ``focus'' constructs in this system to compare its expressiveness to
%% Alms. The ``adopt'' rule can be encoded similarly to the ShareRef
%% example~\cite{Mazurak:2010:LLT}. It is not immediately clear how to encode
%% the ``focus'' rule within the present semantics, as one needs to prevent
%% non-linear access to its argument implying the presence of knowledge about
%% aliasing within the typing judgements; absence of aliasing is not
%% identified with linear kinded types in \fpop.

\citeauthor{??} extend \citeauthor{Wadler:2014}'s work to support polymorphism
in GV. (polymorphism in CP) Future work is to support reduction under these
terms.
