\section{Future Work}

\textbf{CP-GV system.} The issues regarding cut elimination suggest there
should be greater study of calculi based on the sequent calculus. It is
unclear to what extent process calculi such as CP and $\lambda$ calculi can be
unified to use the same framework or library. The different considerations of
sequent and process calculi perhaps mandate a different approach for each. It
would be interesting to consider CP in the setting described by
\citeauthor{Tews:2013} and \citeauthor{Dawson:2010}, where explicit
representation is given to derivations. The aim here would be to make progress
on proving the cut elimination theorem for CP, but it is unclear how linearity
and permutation of environments would be handled in this setting. Otherwise,
defining an appropriate termination measure for the current version of CP may
be possible, though it remains unclear how to obtain the desired relationship
between subprocesses of cut (as explained in \S~\ref{sec:cutelim}).

Further progress with the graph relation version of the translation could be
made in the current setting aiming to limit the duplication discussed
previously. The translation would be substantially affected by the
aforementioned change to CP because the derivations in the conlusions of the
rules would have to be constructed. In this case, it may be worth
investigating representing derivations in GV explicitly.

\citeauthor{Lindley:2014:SAP} extend \citeauthor{Wadler:2014}'s work to
support polymorphism in GV and $n$-ary branch and choice in CP. Future work is
to support these constructs.

\textbf{Beyond Linear.} Most session-based type systems treat channels as
linear resources which may not be aliased. The ``adoption and focus''
system~\cite{Fahndrich:2002} provides a notion of aliased access to linear
resources. The ``adopt'' permits aliases on linear resources and ``focus'' is
used to access a linear resource through one of the aliases. The type system
requires special language features to manage
capabilities. \citeauthor{Fahndrich:2002} do not provide operational semantics
or soundness results for the type system. It would be interesting to
investigate these properties formally in a session type setting. The ``focus''
rule enforces an invariant on the focussed object (channel) such that the type
is maintained during the focus operation. The object may be aliased while it
is under a focus operation. Such a rule would appear to only benefit recursive
session types. Indeed, focussing on a non-recursive channel would appear
pointless because one is prevented from communicating across it for the entire
scope of the focus. Moreover, it would be undesirable to permit modification
of a session type without guaranteeing exclusive access to the channel since
this could lead to invalid communication.

\begin{comment}
 \begin{gather*}
  \prooftree
    \begin{array}{l}
      \afsequent{\Delta}{\Gamma}{C}
                {\typedid{e_1}{\Tguard{G}{h}}}
                {\tensor{C_1}{C_2}} \\
      \capsequent{\Delta}{C_1}{G} \\
      \rho~\mbox{fresh} \\
      \afsequent{\Delta}{\Gamma[\typedid{x}{\AFtr{\rho}}]}
                {\tensor{C_2}{\AFcap{\rho}{h}}}
                {\typedid{e_2}{\tau_2}}
                {\tensor{C_3}{\AFcap{\rho}{h}}}
    \end{array}
  \justifies
    \afsequent{\Delta}{\Gamma}{C}
              {\typedid{\LetFocus{x}{e_1}{e_2}}{\tau_2}}
              {\tensor{C_1}{C_3}}
  \using{\mbox{[focus]}}
  \endprooftree
\end{gather*}


The typing judgements include \emph{capabilities} that track aliasing to heap
objects, e.g. $\AFcap{\rho}{h}$ denotes a mapping from a static name $\rho$ to
a heap type $h$, and a variable $x$ may be typed as $\typedid{x}{\AFtr{\rho}}$
indicating that $x$ refers to an object of type $h$. The purpose of the
capabilities is to enforce restrictions on when an object can be
referenced. In other words, if $\AFcap{\rho}{h}$ is in the capabilities at
some program point then $x$ may be used, otherwise such use is prohibited. In
the [focus] rule, evaluating $e_1$ retrieves the object to be focussed. Note
the type of $e_1$ is a ``guarded'' type. The guard $G$ is the static name of
some object which ``adopted''~\footnote{I omit the adopt typing rule for
  brevity.} an object of heap type $h$. If a program point has
$\AFcap{G}{h_1}$ in the capabilities for some $h_1$ then all objects guarded
by $G$ may be used; $\capsequent{\Delta}{C_1}{G}$ denotes containment of a
such a capability in $C_1$. The guarded type may be thought of as policing
manipulation of linear components within $h$. Treating $\otimes$ as the
disjoint union for capability sets, the typing for $e_2$ temporarily removes
the guard $G$ from the set of capabilities, and adds a new variable $x$, and
associated capability, with a type permitting access to the object of type $h$
i.e. a non-guarded type. Removing $G$ will prohibit access to allow objects
guarded by $G$ for the during of $e_2$. The final capabilities after executing
$e_2$ are required to be $\tensor{C_3}{\AFcap{\rho}{h}}$, enforcing that the
type must remain as $h$. The typing of the $\letterm$ expression ensures that
the guard $G$ (contained in $C_1$) is restored.
\end{comment}
