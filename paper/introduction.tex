\section{Introduction}\label{sec:intro}

Distributed communication systems are becoming increasingly prevalent in
modern society. To facilitate communication between concurrently executing
agents requires complex network protocols to define the type and sequencing of
messages. Typically these protocols are implemented in systems programming
languages, such as C, which provide no correctness guarantees of the
implementation. Furthermore, programming in these languages is often extremely
error-prone, as one has a myriad of unsafe features at one's disposal. As more
businesses begin to conduct transactions over the Web, it is imperative that
they can reason about the communication protocols upon which they rely.

Encoding communication protocols as \textit{session types} is one approach to
achieve guarantees about correctness. One particular variant of session types
are \textit{binary} session types~\footnote{There are also \textit{multiparty}
 session types, but I do not discuss them here.}. A binary session type
specifies communication between two agents, and it specifies the order and
type of messages to be sent or received by an agent in the system. The two
agents have dual session types such that, for example, when one is expecting
to receive a message the other is expecting to send a message; both session
types agree on the type of message to be exchanged. In this way, correctness
of communication can be statically assured by providing communicating agents
with a \textit{channel} whose endpoints have dual session types. A channel is
defined as a bidirectional conduit for exchanging messages, not unlike a
socket in network programming. In contrast to sockets however, a channel is
associated with an explicit type which provides a description of its
protocol. Thus, in session-based systems a channel is usually provided as a
primitive of the language.

In programming language research, work on static type systems is usually
presented on paper complete with informal proofs of correctness for the system
properties. Most types systems in the session types community have been
developed in this way. However, it is more difficult to extend work using this
approach, as it may be unclear how changes affect proofs and one will
typically have to reconsider all proofs to be convinced the properties still
hold. More generally, unassisted formalisms are error-prone to develop and do
not provide a firm basis upon which other researchers can experiment.

Proof assistants such as Coq~\cite{Coq:manual} offer a platform on which to
develop programming languages, type systems and associated proofs of
properties with strong guarantees of correctness. Not only is proving with a
proof assistant much less error-prone than the unaided approach, it also eases
extension of the system since the proof assistant will highlight parts of
proofs that have been affected by changes. On the other hand, one may have to
prove more properties using this approach since this setting does not permit
appealing to human intuition. For example, alpha-equivalence and freshness of
binders are usually assumed in informal presentations, whereas in the
mechanised setting one has to explicitly encode and reason about these
properties~\cite{Aydemir:2005:MMM, Aydemir:2008:EFM}.

This project describes work towards the formalisation of calculi developed by
\citeauthor{Wadler:2014}~\cite{Wadler:2014} describing a relationship between
classical linear logic (CP) and a functional language with session types (GV)
using interactive theorem proving (in particular, the system Coq). The
formalisation is a contribution to the practice of programming language
metatheory and informs what is required of a framework for formalising
$\lambda$-calculus type systems with session types. Such a framework should
enable a whole class of related languages and type systems to be formalised
with little extra proof effort and is inspired by previous work on a
mechanised framework for $\pi$-calculus type systems~\cite{Gay:2001:FFP}. The
current work focuses on the $\lambda$-calculus to provide a basis on which
session types for mainstream programming languages can be studied, taking
advantage of previous proof effort through the reuse of existing libraries for
programming language metatheory. This work has presented a number of
interesting challenges for the formalisation of programming languages coupled
with process calculi, and as a result generated a number of useful guidelines
and issues to consider.
