\section{Related Work}\label{sec:rw}

The background for this project can be categorised roughly as previous work on
session-based type systems (sub-divided into $\pi$ and $\lambda$-calculi
variants, \S \ref{sec:pis} and \S \ref{sec:lam}, respectively) and previous
work on formal verification using proof assistants (particularly focussed on
programming language mechanisation). Finally, some recent work has merged
these two disciplines by verifying a session-based $\pi$-calculus system (\S
\ref{sec:asts}).

\subsection{Session-based Type Systems and Linear Logic}\label{sec:sts}

\subsubsection{Preamble}\label{sec:pre}

The definition of session types will be important for the remainder:
\begin{align*}
 S \bnf\ & \mbox{\lstinline{end}} \alt \inpt{T}{S} \alt \outpt{T}{S} \alt
    \Branch{l_i}{S_i}{i\in I} \alt \Choice{l_i}{S_i}{i\in I}
\end{align*}


Here, I use a modified version of the definition given by
\citeauthor{Gay:2010:LAST}~\cite{Gay:2010:LAST} where I have omitted recursive
session types. The type \lstinline{end} denotes the terminal session type and
a channel typed as such may not perform any communication. A communication
channel with session type $\inpt{T}{S}$ expects to receive a term of type $T$
and then continues with session type $S$. Conversely, a channel with type
$\outpt{T}{S}$ can send a term of type $T$ and then has type $S$. Thus, input
and output are considered \emph{dual} session types, in the sense that a
channel with one endpoint having type $\inpt{T}{S}$ and the other endpoint
having type $\outpt{T}{\Sbar{S}}$, where $\Sbar{S}$ is the dual of $S$,
permits communication between two agents, allowing one to send a term of type
$T$ and the other to receive it with continuing endpoint types $\Sbar{S}$ and
$S$, respectively. Similarly, the remaining types $\Branch{l_i}{S_i}{i\in I}$
and $\Choice{l_i}{S_i}{i\in I}$ are duals of each other. A channel of type
$\Branch{l_i}{S_i}{i\in I}$ can receive a label $l_i$ which will select the
corresponding session type $S_i$ as the continuing type of the channel. While
a channel with type $\Choice{l_i}{S_i}{i\in I}$ can send a label $l_i$ and
continues as $S_i$.

\subsubsection{In the \texorpdfstring{$\pi$}{pi}-calculus}\label{sec:pis}

\citeauthor{Caires:2010:STI} \cite{Caires:2010:STI} present a connection
between a $\pi$-calculus type system with session types ($\pi$DILL) and dual
intuitionistic linear logic (DILL)~\cite{Barber:1996}. They equate linear
propositions with session types of the process calculus via a bidirectional
translation relation. They use a binary reduction relation and a labelled
transition system to represent process reduction and process action with its
environment as is standard~\cite{Sangiorgi:2001}.
\begin{figure*}
  \begin{gather*}
    \prooftree
    \tsequent{\Gamma}{\Delta,\typedid{x}{A}}{P}{T}\quad
    \tsequent{\Gamma}{\Delta,\typedid{x}{B}}{Q}{T}
    \justifies
    \tsequent{\Gamma}{\Delta,\typedid{x}{\BChoice{A}{B}}}{\CPcase{x}{P}{Q}}{T}
    \using{\SILLchoiceL}
    \endprooftree
    \\
    \prooftree
    \tsequent{\Gamma}{\Delta}{P}{\typedid{x}{A}}
    \justifies
    \tsequent{\Gamma}{\Delta}{\CPinl{x}{P}}
             {\typedid{x}{\BChoice{A}{B}}}
             \using{\SILLchoiceROne}
             \endprooftree
             \quad
             \prooftree
             \tsequent{\Gamma}{\Delta}{P}{\typedid{x}{B}}
             \justifies
             \tsequent{\Gamma}{\Delta}{\CPinr{x}{P}}
                      {\typedid{x}{\BChoice{A}{B}}}
                      \using{\SILLchoiceRTwo}
                      \endprooftree
                      \\
                      \\
                      \prooftree
                      \tsequent{\Gamma}{\Delta}{P}{\typedid{x}{A}}\quad
                      \tsequent{\Gamma}{\Delta}{Q}{\typedid{x}{B}}
                      \justifies
                      \tsequent{\Gamma}{\Delta}{\CPcase{x}{P}{Q}}
                               {\typedid{x}{\BBranch{A}{B}}}
                      \using{\SILLbranch}
                      \endprooftree
                      \\
                      \prooftree
                      \tsequent{\Gamma}{\Delta,\typedid{x}{A}}{P}{T}
                      \justifies
                      \tsequent{\Gamma}{\Delta,\typedid{x}{\BBranch{A}{B}}}
                               {\CPinl{x}{P}}{T}
                      \using{\SILLbranchLOne}
                      \endprooftree
                      \quad
                      \prooftree
                      \tsequent{\Gamma}{\Delta,\typedid{x}{B}}{P}{T}
                      \justifies
                      \tsequent{\Gamma}{\Delta,\typedid{x}{\BBranch{A}{B}}}
                               {\CPinr{x}{P}}{T}
                      \using{\SILLbranchLTwo}
                      \endprooftree
  \end{gather*}
  \caption{Branching rules in $\pi$DILL~\cite{Caires:2010:STI}}
  \label{fig:sill-branching}
\end{figure*}


They represent dual intuitionistic linear logic as a sequent calculus
providing logical connectives which have session-based interpretations in
$\pi$DILL. Consider their rules for branch and choice in $\pi$DILL in
Figure~\ref{fig:sill-branching}, which have only two alternatives
(cf. multiple alternatives in the standard definition in \S~\ref{sec:pre}).

Note that branch and choice both have some notion of branching (R rules for
$\oplus$/L rules for $\with$) and at least for \& this is counter-intuitive in
a session-based setting, since the intuition for \& is that of
\emph{offering} (receiving) a choice between alternatives rather than
\emph{making} (sending) the choice. Here, however, the intuition should be
extended to account for its location in the typing judgement. The channel is
on the left of the sequent, and \citeauthor{Caires:2010:STI} suggest one
consider the context as services being provided to the typed process. Thus,
the type of channels in the context is really the type of the other endpoint
of the channel, and the typed process behaves according to the dual
protocol. For example, considering $\SILLbranchLTwo$, and taking $A^\bot$ to
represent the dual of a protocol $A$ as defined by
\citeauthor{Wadler:2014}~\cite{Wadler:2014}, the typed process is behaving
according to the protocol $\BChoice{A^\bot}{B^\bot}$, selecting from
alternatives (in this case, $\inrterm$) and then behaving according to the
selected protocol (in this case, protocol $B^\bot$).

As noted by Wadler~\cite{Wadler:2014} there is a certain amount of redundancy
in the two-sided nature of the rules forcing one to commit to a certain
convention when typing a communication system. Consider, an equivalent system
to the buy/quote example given by
\citeauthor{Caires:2010:STI}~\cite[\S~3]{Caires:2010:STI}:
\begin{align*}
ClientProto \triangleq & (\tensor{Book}
                                 {(\tensor{Card}{(\lfn{Receipt}{\1})})})\\
                       & \oplus\\
                       & (\tensor{Book}{(\lfn{Quote}{\1})})
\end{align*}


Note, I am using $\tensor{}{}$ for session output and $\lfn{}{}$ for session
input following the presentation by \citeauthor{Caires:2010:STI}. In the
protocol defined above, a client selects between buying a book or receiving a
price quote for a book. In the buy option case, the client sends the name of
the book to the server and their card details, then the server sends the
receipt for the purchase. In the quote option case, the client sends the name
of the book and receives a price quote from the server. I have replaced the
types $N$ and $I$ used by \citeauthor{Caires:2010:STI} with more meaningful
type names $Book$, $Card$, $Quote$ and $Receipt$ where appropriate; all are
synonyms for the type $\1$ which denotes the terminal session type in this
system. The client and server process bodies are as follows:
\begin{gather*}
ClientBody_c \triangleq
     {\CPinr{c}{\SLoutput{1984}{c}{\SLinput{c}{quote}{\0}}}}\\
\begin{array}{ll}
ServerBody_c \triangleq c.\caseterm(&\SLinput{c}{book}
                                              {\SLinput{c}{card}
                                              {\SLoutput{receipt}{c}{\0}}},\\
                                    & \SLinput{c}{book}
                                                  {\SLoutput{quote}{c}{\0}})
\end{array}
\end{gather*}


\begin{samepage}
The typings for the client and server processes are given by:
\begin{gather*}
\tsequent{\cdot}{\cdot}{ClientBody_c}{\typedid{c}{ClientProto}}
\tag{\mkTrule{Client}} \label{eq:cbtyp}
\\ \tsequent{\cdot}{\typedid{c}{ClientProto}}{ServerBody_c}{\typedid{-}{\1}}
\tag{\mkTrule{Server}} \label{eq:sbtyp}
\end{gather*}
\end{samepage}

The derivation for \eqref{eq:cbtyp} is $T\1 R$, $\lfn{T}{R}$, $\tensor{T}{R}$,
$\BChoice{T}{R_2}$ and \eqref{eq:sbtyp} is derivable from
$\BChoice{T}{L}$. Except for the final rule, these are exact dual rules of the
rules used in example by \citeauthor{Caires:2010:STI}. Indeed, consider the
composite process:
\begin{gather*}
\sdef{QSimple}{\Spar{c}{ClientBody_c}{ServerBody_c}}\\
\tsequent{\cdot}{\cdot}{QSimple}{\typedid{-}{\1}}
\end{gather*}

The final system is near identical to the one produced in the example given by
\citeauthor{Caires:2010:STI} except the individual processes within the
parallel composition have been swapped. Here the client communicates along the
channel $c$ on the right-side of the sequent and the server utilises the
channel $c$ on the left of the sequent. This typing, while correct, violates
the intuition given by \citeauthor{Caires:2010:STI} of the context providing
the services used by the process, since one would usually associate provision
of services with the server not the client. Instead, the example above
suggests the server utilises some channel provided by the client. On the other
hand, one could imagine such a channel being setup between two communicating
agents; the ``owner'' of the channel is largely irrelevant. Note that
$ClientProto$ is in fact the dual of the $ServerProto$ definition by
\citeauthor{Caires:2010:STI} using the definition of dual used in Wadler's
classical process calculus~\cite{Wadler:2014}. In $\pi$DILL, the duality
between endpoints of the same channel are obscured by the nature of the typing
judgements.

Process reduction steps are equated with proof reduction via cut
elimination. The relationship with proof normalisation in DILL ensures
freedom from deadlock in session-based communication for the process
calculus. The correspondence of session types to linear logic propositions is
an important step in providing a solid logical foundation for a concurrent
type theory; in the same way ``sequential'' type theory ($\lambda$-calculi)
profitted from a logical perspective. However, for session types to become a
more usable tool for programming distributed communication systems (the
ultimate goal, of course, is to write more correct programs), one cannot be
expected to program directly in the $\pi$-calculus; a more mainstream
programming paradigm must be built upon these solid logical foundations. With
that in mind, enter the $\lambda$-calculus!

\subsubsection{In the \texorpdfstring{$\lambda$}{lambda}-calculus}
\label{sec:lam}

A series of works by \citeauthor{Gay:2003:STI}~\cite{Gay:2003:STI} and
\citeauthor{Vasconcelos:2006:TCM}~\cite{Vasconcelos:2006:TCM} provide a
$\lambda$-calculus based language with session types for channel
primitives. The type system is nonstandard, featuring an additional
environment for tracking the session types of channels; these types are
modified during term reduction to reflect communication. The extra channel
environment is provided to support possible aliasing of channels by treating
channel variables as pointers or references into the channel environment. So,
for a channel $c$ of session type $S$, a variable $x$ referring to this
channel would be declared to have type $Chan~c$ and $c~:~S$ would be in the
channel environment. The typing judgements and term reduction relation must be
augmented to handle these environments. Unfortunately, a side-effect of this
is that function types must specify the pre and post-states of channel
environments and therefore, the type of arguments cannot work up to
alpha-equivalence of channel identifiers. That is, if two channels $c$ and $d$
have the same session type, one cannot define a function to operate on
variables of either channel type since the function definition will explicitly
mention the identifier for the channel (in this case either, $Chan~c$ or
$Chan~d$). However, the system does provide an attractive non-linear channel
type which has been neglected in favour of a linear treatment of channels in
more recent
works~\cite{Gay:2010:LAST,Mazurak:2010:LCC,Wadler:2014}. Certainly, it is the
case that an aliased approach would fit better with mainstream programming
languages and perhaps studying a similar system could offer some insights.

The Alms language~\cite{Aldrich:2009} provides a functional language with
affine types based on linear logic. A notion of type \textit{kind} is used to
distinguish between affine and unrestricted values. The typing judgements use
two contexts; one for variables with unrestricted kind and another for
variables of affine type and type variables. Functions are also labelled with
a kind which may depend on their arguments e.g. a multiple argument function
must have affine kinding if it accepts an affine typed resource as an argument
other than the final argument. The reason for this is simply to prevent the
function from using an affine resource more than once. Additionally, functions
act as closures over their environments so their qualifier may be determined
by a free variable in the environment, and a relation is provided to calculate
the appropriate kind qualifier. Type variables are provided to support kind
polymorphism of function qualifiers. The system is expressive enough to
implement session types as a module using a similar approach to
\citeauthor{Gay:2010:LAST}~\cite{Gay:2010:LAST} although treating channels as
affine rather than linearly typed.

%% \citeauthor{Aldrich:2009} note that Alms' type system can support the
%% ``adoption and focus'' system but do not specify details or whether any
%% changes are necessary to their typing judgements.

The work on a linear functional language with session types by
\citeauthor{Gay:2010:LAST}~\cite{Gay:2010:LAST} is the main influence for the
current project from the body of work on session-based $\lambda$-calculus type
systems. They provide paper proofs of their system's type safety. Given the
non-trivial features of the type system such as recursive session types and
buffered channels for supporting asynchronous communication it will be
challenging to extend these proofs manually. A mechanised version would
provide stronger guarantees of correctness and allow others to more easily
alter proofs as a result of changes made to the language.

\subsubsection{Fusion}

A number of systems incorporate both a functional language surface syntax with
a translation to a more primitive process calculus. I shall focus on two such
systems presented recently which have a connection to linear logic.

Lolliproc~\cite{Mazurak:2010:LCC} provides a functional language with
concurrency primitives as a surface syntax on top of an underlying process
calculus. The process calculus provides a classical linear logic
interpretation for the functional language, utilising double negation
elimination to provide session type duality. The surface syntax is an
effective abstraction layer for the programmer since it does not expose the
underlying process calculus.

Concurrency is provided by control operators which, respectively, spawn a
child process and wait for a child process to complete. The spawning process
creates a channel between parent and child process; a channel is represented
as a continuation and the type denotes the communication protocol
(i.e. session type) between parent and child.

\citeauthor{Mazurak:2010:LCC} present proofs for type safety, strong
normalisation and confluence. Confluence provides the guarantee that no race
conditions can occur, strong normalisation prohibits non-terminating programs
and type safety assures deadlock-freedom. Deadlock-freedom is implied by the
acyclic communication graphs (progress of process reduction) which occur
between parallel processes; follows from permitting only one channel between
two halves of parallel composition construct in the process calculus. Most of
the development is supported by Coq proof scripts with remaining issues due to
reasoning about communication graphs. The development does not provide
primitive send and receives; instead, these are encoded as linear functions.
%% and classical extension to second-order logic by
%% \citeauthor{Mazurak:2013:LPP}'s PhD thesis~\cite{Mazurak:2013:LPP}.

\citeauthor{Wadler:2014}'s recent work~\cite{Wadler:2014} brings together a
process calculus, CP, based on the earlier work by
\citeauthor{Caires:2010:STI} with a functional language, GV, based on LAST by
\citeauthor{Gay:2010:LAST}. GV presents session types using the standard
presentation as described in \S~\ref{sec:pre}. CP presents linear logical
connectives with session type interpretations inspired by
\citeauthor{Caires:2010:STI}~\cite{Caires:2010:STI}. \citeauthor{Wadler:2014}
describes a continuation-passing style (CPS) translation from GV session types
to CP classical linear logic propositions. The process calculus CP is slightly
different to $\pi$DILL by \citeauthor{Caires:2010:STI}, apart from being
classical in nature, the constructs for server replication are modified; CP
provides a weakening and contraction rules whereas these are encoded in one
rule for process reduction by \citeauthor{Caires:2010:STI}. Additionally,
rather than a two-sided sequent calculus, \citeauthor{Wadler:2014} uses
one-sided sequents for defining the process calculus, which provides a more
intuitive presentation of session type duality and clearer correspondence to a
$\lambda$-calculi surface syntax, simplifying the translation presentation. As
presented, GV does not support all features of CP and thus the translation is
one way (GV to CP). \citeauthor{Lindley:2014:SAP}~\cite{Lindley:2014:SAP}
extend GV to provide support for polymorphism and replication, following a
similar CPS translation scheme. Lastly,
\citeauthor{Lindley:2014:SPS}~\cite{Lindley:2014:SPS} describe operational
semantics for GV and their system has greater similarities with LAST than
prior GV works, for example, by treating send and receive as linear functions
rather than language primitives.

\subsection{Mechanising PL Metatheory}

Interactive theorem proving and proof assistants have seen an upsurge in use
in recent years. Especially within the domain of programming language
research, there is a strong emphasis on providing mechanised proofs
accompanying work in the area. Indeed, \citeauthor{Aydemir:2005:MMM} have
published a series of challenges on mechanising programming languages aimed at
providing a starting point for comparing different representation
techniques~\cite{Aydemir:2005:MMM}. Another aim is to provide reusable
libraries for common reasoning needed across programming language
developments, e.g. handling of typing environments. This effort has resulted
in a number of different approaches for representing programming languages
within proof assistants. For instance,
\citeauthor{Aydemir:2008:EFM}~\cite{Aydemir:2008:EFM} present a representation
of $\lambda$-calculi where bound variables are represented as \textit{de
  Bruijn indices}~\footnote{A de Bruijn index is a number used to represent a
  bound variable which indicates the position of its binder (starting from
  zero for the innermost binder). For example, $\lambda~(\lambda~1)$
  corresponds to $\lambda x. \lambda y. x$, i.e. the constant function.} and
free variables are represented as named terms; the \textit{locally nameless}
representation. Their work resulted in a library for handling commonly
occurring aspects in programming language metatheory which has been used since
in other developments~\cite{Park:2014:MMW}. Reusable libraries reduce the
proof engineering effort needed for later works, in contrast to the
pen-and-paper approach, and some proof assistants (as is the case with Coq)
allow one to extract a program from the development which can provide a
typechecker or compiler for the language formalised.~\footnote{Curry-Howard
  strikes again; a proof of decidability of typechecking corresponds to a
  program implementing a typechecker for the language.}

Some recent mechanisation effort focuses on providing a basis for studying
linearity within type systems. This work is of interest to the current project
since most session-based type systems assume a strictly linear type system
which requires re-binding of channel identifiers, such as in the functional
setting of \citeauthor{Gay:2010:LAST}. \citeauthor{Mazurak:2010:LLT} present
an extension to System F (termed \fpop) with a notion describing types as
having linear or non-linear \textit{kind}~\cite{Mazurak:2010:LLT}. This
extension is motivated by previous work on incorporating linearity into type
systems. Overall, previous attempts either hamper the inclusion of desirable
programming features (such as polymorphism), or do not adequately reflect
non-linearity as the common case leading to awkward programming. \fpop avoids
these deficiencies by categorising types into kinds whilst maintaining similar
semantics to System F. Mechanised proofs for type soundness and parametricity
are presented, and the semantics allow non-linear types to be treated as
linear. Examples show the system can provide a wide range of permissions on
interfaces, from full exclusive access to read-only shared references. The
system is very close to the Alms language described by
\citeauthor{Aldrich:2009} but has the advantage that the soundness results
(type safety and parametricity) have been entirely mechanised in Coq.

%%  It would be interesting to see if one could encode the ``adopt'' and
%% ``focus'' constructs in this system to compare its expressiveness to
%% Alms. The ``adopt'' rule can be encoded similarly to the ShareRef
%% example~\cite{Mazurak:2010:LLT}. It is not immediately clear how to encode
%% the ``focus'' rule within the present semantics, as one needs to prevent
%% non-linear access to its argument implying the presence of knowledge about
%% aliasing within the typing judgements; absence of aliasing is not
%% identified with linear kinded types in \fpop.

While not related to programming language metatheory,
\citeauthor{Gay:2001:FFP}~\cite{Gay:2001:FFP} provides a $\pi$-calculus
framework mechanised in Isabelle/HOL. The aim is to build a mechanisaton of
session-based type systems on top of the $\pi$-calculus framework. The use of
de Bruijn indices for both bound and free variables causes issues with
variable substitution often requiring permutation of typing environments (see
\citeauthor{Aydemir:2008:EFM}~\cite{Aydemir:2008:EFM} for a discussion on
binder representations) and lifting of free variables during substitution; I
avoid this issue in my representation choice (\S \ref{sec:approach}).

The introduction of names for hypotheses during execution of custom tactics
are an issue in the proof development. These names are automatically generated
by the Isabelle/HOL system which creates a dependency between the names chosen
and the tactic. Thus, one must be careful to use the exact names expected by
the tactic. In the Coq proof assistant, defining custom tactics can be
achieved using the Ltac language~\cite{Delahaye:2000:TLS}. Among other things,
Ltac supports pattern matching on hypotheses and goal forms. Thus, if one
wished to pattern match on an inductive type describing variables, one could
do so without mentioning the actual variable name as follows:

\begin{coq}
Ltac my_tactic :=
  match goal with
  | [v: var |- _] =>
    ... tactics here possibly mentioning `v' hypothesis ...
  end.
\end{coq}

\subsubsection{Applications to Session-based Type Systems}\label{sec:asts}

\citeauthor{Goto:2014}~\cite{Goto:2014} provide a $\pi$-calculus session-based
type system providing session polymorphism. The system is more general than
session polymorphism via subtyping~\cite{Gay:2005:SST} in that it is capable
of typing a generic forwarding process between mutually dual session
types. Defining such a process is not possible using subtyping since one
wishes to restrict the types to be dual, not a subtype of some other
type. This more general form of polymorphism is achieved by permitting input
on all session types (including \lstinline{end}, the terminal session type)
and then employing transition hypotheses to ensure the type permits such
input. The authors mechanise their type system and its properties in the Coq
proof assistant. However, the mechanised system is restricted to the
$\pi$-calculus and the main focus is proving soundness properties for the
deduction rules allowing polymorphism, rather than a more general basis for
studying programming with session types.
