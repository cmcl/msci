\documentclass{beamer}

\usepackage[latin1]{inputenc}
\usepackage{graphicx}
\usepackage{subfigure}
\usepackage{amsmath}
\usepackage{listings}
\usepackage{comment}
\usepackage{tikz}
\usetikzlibrary{arrows,positioning}
\usepackage[style=alphabetic,natbib=true]{biblatex}

\usetheme{Warsaw}

\title[Project Presentation]{Propositions as Sessions, Mechanically}
\author{Craig McLaughlin}
\institute{University Of Glasgow}
\date{\today}

\begin{document}

\begin{frame}
\titlepage
\end{frame}


today i am going to talk about a formalisation effort based on an existing
type system whose properties have previously only been studied
informally. first in order to motivate the previous and current work i shall
make two general observations here. one: distributed communication systems are
becoming increasingly prevalent in modern society. two: the correct
functioning of these systems hinges on the protocol between any two
communicating agents to be well-defined and correctly implemented.

\begin{frame}
Session types by example
\end{frame}

one approach to achieve such correctness guarantees is by using binary session
types. [INTRODUCE BINARY SESSION TYPES HERE]. specifying a communication
protocol then involves providing mutual dual types to the communicating
parties. [a simple example with two processes is displayed].

\begin{frame}
Hand-written (informal) proofs $=$ Error-prone $+$ Difficult to update
\end{frame}

there has been a lot of activity in this area producing a variety of different
session-based type systems. many systems build on top of previous work adding
more complex features such as asynchronous, polymorphic and recursive session
types. extending informal proofs is error-prone and difficult because it is
not always clear which parts of a system's metatheory is affected by a
change. a formalised system would reduce likelihood of errors in the proofs
and aid extension since the theorem prover would highlight changes to the
metatheory

\begin{frame}
CP - process calculus with operators from classical linear logic
GV - functional language with session type
CP -> GV translation (interprets classical linear logic operators as session
types)
\end{frame}

the formalisation i present here is based on recent work providing a logical
foundation for understanding session types. in particular i present work
towards a formalisation of Wadler's original CP/GV system which describes a
two-tier type system with session types. A high-level overview of the system
is as follows: we have a process calculus CP with operators from classical
linear logic, a high-level functional language GV with session types, and a
translation from GV to CP which interprets operators in classical linear logic
as session types.

\begin{frame}
an example of CP classical linear logic operators and their session type
interpretation
\end{frame}

by classical linear logic i mean a logic where negation is an involution which
restricts the use of certain rules on resources, namely that all linear
resources must be used (corresponding to prohibiting weakening) and that one
cannot duplicate a linear resource (corresponding to contraction). some
operators in the CP calculus are presented. the interpretation of tensor
product is output of A and continue as B. The dual of this is input of A and
continue as the negation of B. Thus negation in classical linear logic
corresponds to session duality.

\begin{frame}
an example of GV and a translation???
\end{frame}

GV has built-in support for session types. consider the rule for sending a
value of type T along a channel N of type !T.S. the translation to CP is
presented. blah blah something about the session types being a little odd or
overlook this probably

\begin{frame}
- subject reduction of CP
- top-level cut elimination
- translation preserving well-typedness
\end{frame}

the system has three main properties. subject reduction of CP, top-level
cut-elimination and translation preserving well-typedness. One out of the
three has been formalised and some issues and guidelines have been developed
accordingly. some comments on progress towards the other two is offered

The system
enjoys freedom from deadlock cut elimination in CP corresponds to
communication in GV.

the formalisation i will describe today regards this
connection, I will present work towards a formalisation of the system GV/CP
originally described by Phil Wadler was the aim of this project to mechanise
it. it provides a basis for further additions and also provides some
properties that may be desirable such as deadlock freedom. the formalisation
provides a means to gain more confidence in the proofs but also assists in
extending them to account for the effects of additions.

\begin{comment}
whats the problem
what have others done
why are we doing it
what did we do
what have we learned
\end{comment}
\begin{frame}
\frametitle{Preamble}
\item Propositions as Sessions
\item system features
\item Coq
\item encoding
\item subject reduction theorem
\item difficulties in formalisation
\item implications
\item future work
\end{frame}

\printbibliography

\end{document}
