\documentclass{beamer}

\usepackage[latin1]{inputenc}
\usepackage{graphicx}
\usepackage{subfigure}
\usepackage{amsmath}
\usepackage{listings}
\usepackage{comment}
\usepackage{tikz}
\usetikzlibrary{arrows,positioning}
\usepackage[style=alphabetic,natbib=true]{biblatex}

\usetheme{Warsaw}

\title[Project Presentation]{Propositions as Sessions, Mechanically}
\author{Craig McLaughlin}
\institute{University Of Glasgow}
\date{\today}

\begin{document}

\begin{frame}
\titlepage
\end{frame}

\begin{frame}
Hand-written (informal) proofs $=$ Error-prone $+$ Difficult to update
\end{frame}

today i am going to talk about a formalisation effort based on an existing
type system whose properties have previously only been studied
informally. first in order to motivate the previous and current work i shall
make two general observations here. one: distributed communication systems are
becoming increasingly prevalent in modern society. two: the correct
functioning of these systems hinges on the protocol between any two
communicating agents to be well-defined and correctly implemented.

one approach to achieve such correctness guarantees is by using binary session
types. [INTRODUCE BINARY SESSION TYPES HERE]. specifying a communication
protocol then involves providing mutual dual types to the communicating
parties. [a simple example with two processes is displayed].

there has been a lot of activity in this area producing a variety of different
session-based type systems. many systems build on top of previous work adding
more complex features such as asynchronous, polymorphic and recursive session
types. extending informal proofs is error-prone and difficult because it is
not always clear which parts of a system's metatheory is affected by a
change. a formalised system would aid extension and reduce likelihood
of errors in the proofs since the theorem prover would highlight the effects

recent work has provided a logical foundation for understanding session
types. Wadler's work on the CP/GV system describes a two-tier type system with
session types. [present it on screen.what?] the process calculus CP interprets
linear logical connectives as session types and the surface functional
language GV can be translated into CP. The system enjoys freedom from
deadlock. surface language.

session types correspond to classical linear logic connectives. where negation
in the logic corresponds to the dual property of a session type. [present
  example slide of the connectives in classical linear logic and their session
  type interpretation]. the formalisation i will describe today regards this
connection, I will present work towards a formalisation of the system GV/CP
originally described by Phil Wadler was the aim of this project to mechanise
it. it provides a basis for further additions and also provides some
properties that may be desirable such as deadlock freedom. the formalisation
provides a means to gain more confidence in the proofs but also assists in
extending them to account for the effects of additions.

\begin{comment}
whats the problem
what have others done
why are we doing it
what did we do
what have we learned
\end{comment}
\begin{frame}
\frametitle{Preamble}
\item Propositions as Sessions
\item system features
\item Coq
\item encoding
\item subject reduction theorem
\item difficulties in formalisation
\item implications
\item future work
\end{frame}

\printbibliography

\end{document}
