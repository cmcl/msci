\begin{abstract}
Session types ensure that communicating agents adhere to an agreed upon
protocol. The soundness proofs of session-based type systems are typically
developed by hand and checked manually. The correctness of these proofs is
crucial to maintain the assurances provided by session types. However,
informal proofs are error-prone to develop and extend. The current paper
describes work towards a formalisation of an existing two-tier session-based
type system connecting classical linear logic and session types. The
formalisation, developed in Coq, proves subject reduction for one of the tiers
and makes strides towards the other system properties. This work illustrates
the limitations of state-of-the-art techniques for formalising programming
language metatheory and identifies a number of issues and guidelines of which
practitioners in the field may wish to take account. The paper concludes by
highlighting some areas of future work.
\end{abstract}
