\begin{abstract}
Session types ensure that communicating agents adhere to an agreed upon
protocol. Recent work has presented a number of related systems with
increasingly elaborate session type features including asynchronous buffered
communication, type polymorphism and recursion. That these features are sound
is crucial to maintain the assurances provided by session types. However, the
metatheory of these systems has not been formally studied and proofs are
typically developed by-hand and checked manually. The current paper describes
work towards a formalisation of such a type system based on recent work on a
two-tier system connecting classical linear logic and session types. The
formalisation, developed in Coq, proves subject reduction for one of the tiers
and makes strides towards the other system properties. This work illustrates
the limitations of the state-of-the-art techniques for formalising programming
language metatheory and identifies a number of issues and guidelines of which
practitioners in the field may wish to take account. The paper concludes by
highlighting some areas of future work.
\end{abstract}
