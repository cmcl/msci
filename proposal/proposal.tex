\documentclass{mprop}

\usepackage{graphicx}
\usepackage[style=alphabetic,natbib=true]{biblatex}
\addbibresource{proposal}

% alternative font if you prefer
%\usepackage{times}

% for alternative page numbering use the following package
% and see documentation for commands
%\usepackage{fancyheadings}


% other potentially useful packages
%\uspackage{amssymb,amsmath}
%\usepackage{url}
%\usepackage{fancyvrb}
%\usepackage[final]{pdfpages}

\begin{document}

%%%%%%%%%%%%%%%%%%%%%%%%%%%%%%%%%%%%%%%%%%%%%%%%%%%%%%%%%%%%%%%%%%%

\title{Formalisation of Session-based Lambda Calculus Type Systems}
\author{Craig McLaughlin}
\date{\today}
\maketitle

%%%%%%%%%%%%%%%%%%%%%%%%%%%%%%%%%%%%%%%%%%%%%%%%%%%%%%%%%%%%%%%%%%%

%%%%%%%%%%%%%%%%%%%%%%%%%%%%%%%%%%%%%%%%%%%%%%%%%%%%%%%%%%%%%%%%%%%
\tableofcontents
\newpage
%%%%%%%%%%%%%%%%%%%%%%%%%%%%%%%%%%%%%%%%%%%%%%%%%%%%%%%%%%%%%%%%%%%

%%%%%%%%%%%%%%%%%%%%%%%%%%%%%%%%%%%%%%%%%%%%%%%%%%%%%%%%%%%%%%%%%%%
\section{Introduction}\label{intro}

briefly explain the context of the project problem

\subsection{A subsection}
Please note your proposal need not follow the included section headings - this
is only a suggested structure. Also add subsections etc as required

%%%%%%%%%%%%%%%%%%%%%%%%%%%%%%%%%%%%%%%%%%%%%%%%%%%%%%%%%%%%%%%%%%%
\section{Statement of Problem}

clearly state the problem to be addressed in your forthcoming project. Explain
why it would be worthwhile to solve this problem.

The aim of this project is to develop a framework for formalising lambda
calculus type systems involving session types using interactive theorem
proving (in particular, the system Coq). The motivation for the framework is
that it will enable a whole class of related languages and type systems to be
formalised with little extra proof effort. Inspired by previous work on a
mechanised framework of pi calculus type systems~\cite{Gay:2001:FFP}, we shift
to the lambda calculus to provide a basis on which session types for
mainstream programming languages can be studied whilst taking advantage of
previous proof effort and specialising only for those parts unique to the
particular system.

From the generic framework, a number of directions can be explored including
session polymorphism, session subtyping, and permitting different forms of
aliasing within the type system. The majority of previous session type systems
for mainstream programming languages has focussed exclusively on linear type
systems. One disadvantage of a linear type system is the need to re-bind
objects after each operation~\cite{Gay:2010:LTT}. Principally, we intend to
instantiate the framework with a type system that provides the benefits of
aliasing with the guarantees of linearity, termed ``adoption and focus''. The
``adoption and focus'' type system is non-trivial requiring special language
features, e.g. to manage capabilities which can be thought as a form of object
liveness mechanism, so the framework design must support such flexibility by,
e.g allowing the typing judgements to be extended such that the capabilities
can be included in soundness results. The instantiation will demonstrate that
the framework supports a wide range of languages and type systems providing
researchers the ability to harness the power of formal verification in the
design, implementation and documentation of session-based type systems.

%%%%%%%%%%%%%%%%%%%%%%%%%%%%%%%%%%%%%%%%%%%%%%%%%%%%%%%%%%%%%%%%%%%
\section{Background Survey}

present an overview of relevant previous work including articles, books, and
existing software products. Critically evaluate the strengths and weaknesses
of the previous work.

System F-pop, session polymorphism formalised in Coq, Simon's Isabelle/HOL
development.

%%%%%%%%%%%%%%%%%%%%%%%%%%%%%%%%%%%%%%%%%%%%%%%%%%%%%%%%%%%%%%%%%%%
\section{Proposed Approach}

state how you propose to solve the software development problem. Show that
your proposed approach is feasible, but identify any risks.

%%%%%%%%%%%%%%%%%%%%%%%%%%%%%%%%%%%%%%%%%%%%%%%%%%%%%%%%%%%%%%%%%%%
\section{Work Plan}

show how you plan to organize your work, identifying intermediate deliverables
and dates.

I need to perform some formalisation of the system. I should identify the key
issues (linear typing, bindings (as claimed in the literature), etc).

%%%%%%%%%%%%%%%%%%%%%%%%%%%%%%%%%%%%%%%%%%%%%%%%%%%%%%%%%%%%%%%%%%%

\printbibliography

\end{document}
